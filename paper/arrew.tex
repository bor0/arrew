\documentclass{article}

\usepackage{arxiv}

\usepackage{amsmath}

\title{The Arrew theorem prover}

\author{
  Boro Sitnikovski \\
  Skopje, North Macedonia \\
  \texttt{buritomath@gmail.com} \\
}

\begin{document}

\maketitle

\begin{abstract}
Arrew (ARrow REWriter) is a simple theorem prover that allows expressing formal systems and deriving theorems. It uses a simple substitution rule together with set equality to derive theorems. In this paper, we present the mathematical model of Arrew.
\end{abstract}

\keywords{Theorem prover, mathematical model}

\section{Mathematical model}

A formal system is defined by the tuple $F = (R, S, V, T)$ together with the functions $subst_{rule}^n$ and $subst_{thm}^n$ where $R_n \in R$ is a set of rules of $n$-ary arguments, $S$ is a set of string of symbols, $V$ is a set of variables, and $T$ is a set of theorems. A rule $r \in R_n$ is a sequence of string of symbols; it can be roughly interpreted as a function $r_1 \to \ldots \to r_n$. The $n$-th argument represents a conclusion, and the others represent hypotheses.

Let $V_S \subseteq V \times S$ and $X[s/v]$ denote the expression $X$ in which each occurrence of $v$ is replaced with $s$. The following function performs substitution on a rule's hypotheses and conclusion:
$$ subst_{rule}^n(r, V_S) = {
\begin{cases}
subst_{rule}^n(r_1[s/v], \ldots, r_n[s/v], V_S \setminus \{(v, s) \}), & r = (r_1, \ldots, r_n) \land (v, s) \in V_S \\
r & V_S = \emptyset
\end{cases}}
$$

Let $h = (h_1, \ldots, h_{n-1})$ where $\forall i, h_i \in T$. The function $subst_{thm}^n(h, V_S)$ is defined similarly.

\textbf{Theorem derivation}: We say that $t = subst_{rule}^1((r_n), V_S) \in T$ (i.e., $t$ is a theorem) if and only if:
$$subst_{rule}^{n-1}((r_1, \ldots, r_{n-1}), V_S) = subst_{thm}^{n-1}(h, V_S)$$

Note that for $n = 1$ we have $subst_{rule}^0((), V_S) = () = subst_{thm}^0((), V_S)$ i.e. all 1-ary rules are theorems:
$$\forall r, r \in R_1 \to r \in T$$

\section{Example}

Let $R = \{ \{ (\texttt{MI}) \}, \{ (\texttt{Mx, Mxx}) \} \}$, $S = \{ \texttt{M}, \texttt{I}, \texttt{II} \}$, $V = \{ \texttt{x}, \texttt{y} \}$ and $V_S = \{ (\texttt{x, I}) \}$. Note that $R_1 \in R$ implies $\texttt{MI} \in T$. To prove $\texttt{MII} \in T$, we use the rule within $R_2$ and since $(\texttt{x, I}) \in V \times S$, we get that $subst_{rule}^1((\texttt{Mx}), V_S) = \texttt{MI} = subst_{thm}^1((\texttt{MI}), V_S)$. Since the rule's arguments match the theorem's hypotheses, $subst_{rule}^1((\texttt{Mxx}), V_S) = \texttt{MII} \in T$.

\begin{thebibliography}{1}

\bibitem{b1}
Daniel J. Velleman
\newblock How to Prove It: A Structured Approach.
\newblock Cambridge University Press, 2006.

\bibitem{b2}
Douglas Hofstadter
\newblock Godel, Escher, Bach: an Eternal Golden Braid.
\newblock Basic Books, Inc., 1979.

\end{thebibliography}
\end{document}
