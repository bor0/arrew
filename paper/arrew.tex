\documentclass{article}

\usepackage{arxiv}

\usepackage{amsmath}

\title{The Arrew theorem prover}

\author{
  Boro Sitnikovski \\
  Skopje, North Macedonia \\
  \texttt{buritomath@gmail.com} \\
}

\begin{document}
\maketitle

\begin{abstract}
We present the mathematical model of a simple theorem prover that allows the user to define inference rules and axioms and then derive theorems based on them.
\end{abstract}

\keywords{Theorem prover, mathematical model}

\section{Introduction and mathematical model}

Arrew (ARrow REWriter) is a simple theorem prover that allows expressing formal systems and deriving theorems. It uses a simple substitution rule combined with a set equality check to derive theorems.

\textbf{Definition}: A formal system is defined by the tuple $F = (R, S, V, T)$, together with the functions $subst_{rule}^n$ and $subst_{thm}^n$ where $R_n \in R$ is a set of rules of $n$-ary arguments (the $n$-th argument represents a conclusion, and the others represent hypotheses), $S$ is a set of string of symbols, $V$ is a set of variables, and $T$ is a set of theorems.

A rule $r$ is a sequence of string of symbols; it can be roughly interpreted as a function $r_1 \to \ldots \to r_n$.

Further, let $r \in R_n$, $V_S \subseteq V \times S$, and let $X[s/v]$ be the expression $X$ in which each occurrence of $v$ is replaced with $s$. The following function performs substitution on a rule's hypotheses and conclusion:
$$ subst_{rule}^n(r, V_S) = {
\begin{cases}
subst_{rule}^n(r_1[s/v], \ldots, r_n[s/v], V_S \setminus \{(v, s) \}), & r = (r_1, \ldots, r_n) \land (v, s) \in V_S \\
r & V_S = \emptyset
\end{cases}}
$$

Let $h = (h_1, \ldots, h_{n-1})$ where $\forall i, h_i \in T$ i.e. every hypothesis represents a theorem. The following function performs substitution on a theorem's hypotheses:
$$ subst_{thm}^n(h, V_S) = {
\begin{cases}
subst_{thm}^n(h_1[s/v], \ldots, h_n[s/v], V_S \setminus \{(v, s) \}), & h = (h_1, \ldots, h_n) \land (v, s) \in V_S \\
h & V_S = \emptyset
\end{cases}}
$$

\textbf{Theorem derivation}: Consider applying the rule $r = (r_1, \ldots, r_n) \in R_n$, together with the set of substitutions $V_S \subseteq V \times S$ and a sequence of hypotheses $h = (h_1, \ldots, h_{n-1})$ where $\forall i, h_i \in T$. We say that $t = subst_{rule}^1((r_n), V_S) \in T$ (i.e., $t$ is a theorem) if and only if $subst_{rule}^{n-1}((r_1, \ldots, r_{n-1}), V_S) = subst_{thm}^{n-1}(h, V_S)$.

\textbf{Corollary}: For $n = 1$ we have that $subst_{rule}^0((), V_S) = () = subst_{thm}^0((), V_S)$ i.e. all 1-ary rules are theorems: $\forall r, r \in R_1 \to r \in T$.

\textbf{Example}: Let $R_1 = \{ \texttt{MI} \}$, $r_1 = \texttt{(Mx, Mxx)}$ and $R_2 = \{ r_1 \}$; thus $R = \{ R_1, R_2 \}$. Let $S = \{ \texttt{M}, \texttt{I}, \texttt{II} \}$ and $V = \{ \texttt{x}, \texttt{y} \}$. Note that $R_1$ implies $\texttt{MI} \in T$. We will show $\texttt{MII} \in T$ by using $r_1$ - we need to show that $subst_{rule}^1((\texttt{Mx}), V_S) = subst_{thm}^1((\texttt{MI}), V_S)$. Since $\texttt{(x, I)} \in V \times S$, we get that $subst_{rule}^1((\texttt{Mx}), V_S) = \texttt{MI} = subst_{thm}^1((\texttt{MI}), V_S)$. Since the hypotheses match the rule's arguments, we conclude that $subst_{rule}^1((\texttt{Mxx}), V_S) = \texttt{MII} \in T$.

\begin{thebibliography}{1}

\bibitem{b1}
Daniel J Velleman
\newblock How to Prove It: A Structured Approach.
\newblock Cambridge University Press, 2006.

\end{thebibliography}
\end{document}
