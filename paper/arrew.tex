\documentclass{article}

\usepackage{amsmath}
\usepackage{arxiv}
\usepackage{hyperref}

\title{The Arrew theorem prover}

\author{
  Boro Sitnikovski \\
  Skopje, North Macedonia \\
  \texttt{buritomath@gmail.com} \\
}

\begin{document}

\maketitle

\begin{abstract}
Arrew (Arrow Rewriter) is a mathematical system (theorem prover) that allows expressing and working with formal systems. It relies on a simple substitution rule and set equality to derive theorems.
\end{abstract}

\keywords{Theorem prover, mathematical system}

\section{Mathematical system}

A formal system is defined by the tuple $F = (R, V, T)$ together with the functions $subst_{rule}^n$ and $subst_{thm}^n$ where $R_n \in R$ is a set of rules of $n$-ary arguments, $V$ is a set of variables, and $T$ is a set of theorems. A rule $r = (r_1, \ldots, r_n) \in R_n$ is a sequence of string of symbols; it can be roughly interpreted as a function $r_1 \to \ldots \to r_n$, where the $n$-th argument represents a conclusion, and the others represent hypotheses.

Let $S \subseteq V \times T$ denote a set of substitutions, and $X[t/v]$ denote the expression $X$ in which each occurrence of $v$ is replaced with $t$. We define the following function which performs substitution on a rule's hypotheses and conclusion:
$$ subst_{rule}^n(r, S) = {
\begin{cases}
subst_{rule}^n(r_1[t/v], \ldots, r_n[t/v], S \setminus \{(v, t) \}), & r = (r_1, \ldots, r_n) \land (v, t) \in S \\
r & S = \emptyset
\end{cases}}
$$

Let $h = (h_1, \ldots, h_{n-1})$ where $\forall i, h_i \in T$. The function $subst_{thm}^{n-1}(h, S)$ is defined similarly.

For deriving new theorems, we say that $t = subst_{rule}^1((r_n), S) \in T$ (i.e., $t$ is a theorem) if and only if:
$$subst_{rule}^{n-1}((r_1, \ldots, r_{n-1}), S) = subst_{thm}^{n-1}(h, S)$$

Terms and axioms are represented as 1-ary rules; note that for $n = 1$ we have $subst_{rule}^0((), S) = () = subst_{thm}^0((), S)$ i.e. all 1-ary rules are theorems: $\forall r, r \in R_1 \to r \in T$.

\section{Example}

Let $R = \{ \{ \vdash\texttt{MI}, \texttt{I} \}, \{ (\vdash\texttt{Mx}, \vdash\texttt{Mxx}) \} \}$, $V = \{ \texttt{x} \}$. The particular choice of $R_1$ allows us to pick $S = \{ (\texttt{x}, \texttt{I}) \}$; since \texttt{I} is a 1-ary rule, $\texttt{I} \in T$. Similarly, $\vdash \texttt{MI} \in T$. To prove $\vdash\texttt{MII} \in T$, we use the rule within $R_2$ and since $(\texttt{x}, \texttt{I}) \in S$, we get that $subst_{rule}^1((\vdash\texttt{Mx}), S) = \vdash\texttt{MI} = subst_{thm}^1((\vdash\texttt{MI}), S)$. Since the rule's arguments match the theorem's hypotheses, $subst_{rule}^1((\vdash\texttt{Mxx}), S) = \vdash\texttt{MII} \in T$.

\begin{thebibliography}{1}

\bibitem{b1}
Boro Sitnikovski,
\newblock Arrew theorem prover (Python implementation).
\newblock [Online]. Available: \url{https://github.com/bor0/arrew/} (Accessed Jun. 2022)

\bibitem{b2}
Daniel J. Velleman,
\newblock How to Prove It: A Structured Approach.
\newblock Cambridge University Press, 2006.

\bibitem{b3}
Douglas Hofstadter,
\newblock Godel, Escher, Bach: an Eternal Golden Braid.
\newblock Basic Books, Inc., 1979.

\end{thebibliography}
\end{document}
